\section*{Abstract}
\addcontentsline{toc}{section}{Abstract}
%
Momentum resolved electron energy loss spectroscopy (MREELS) can provide a trove of valuable data about a specimen but often requires specialised and proprietary software to analyse. In this work Python methods that were designed for the analysis of MREELS data are explained and tested. This provided results consistent with earlier findings, namely the existence of the dispersionless $3.5$ $eV$, $7.25$ $eV$ and $22$ $eV$ energy loss peaks in the low momentum transfer region of the q-EELS spectral data. The dispersion of the bulk plasmon was also tracked along different crystallographic directions and found to potentially be anisotropic in the high momentum transfer regime.