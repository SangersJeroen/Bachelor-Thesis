\section{Introduction}
Transmission electron microscopes are incredible machines capable of imaging structures at the atomic scale and can provide useful insight into the electronic properties of the sample being imaged.\\
The files produced by the transmission electron microscope are usually in a proprietary format and made to be analysed with specialised software that can often be quite expensive. Therefore this work will introduce some of the methods implemented in an open-source Python library, the methods to be discussed in this work are those that are used to analyse momentum resolved electron energy loss spectroscopy data sets. This work will use a data set procured by imaging a $\gamma$-indium selenide sample in an energy filtered transmission electron microscope set up for diffraction images.\\
The work starts by introducing all the necessary concepts in the theory section \ref{chap:theory} such that his work can be understood without prior knowledge in the field of electron microscopy. In the experimental method section, chapter \ref{chap:exp-meth}, a majority of the methods implemented will be presented and explained. The effect and usefulness of these methods are discussed in the results section, chapter \ref{chap:results}. The work will be discussed, further intentions with respect to the software will be clarified and the final conclusion is presented in the conclusion chapter \ref{chap:conclusion}.





