\section{Conclusion}
\label{chap:conclusion}%
%
This work set out to introduce the reader to existing methods implemented in Python and validate the results with other earlier works.\\
From the data extraction techniques that were applied to the EFTEM stack the least useful one appeared to be the radial integration method since it was unable to isolate data in a single crystallographic direction. The other two extraction techniques are both useful for different purposes, the line integration technique was able to pool EFTEM columns of similar momentum transfer well and thus improved the momentum resolution of the q-EELS map whilst allowing the introduction of some data not truly in line with the crystallographic direction to be analysed. The stack slicing method delivered the 'purest' data but did so at the cost of momentum transfer resolution, the lack of pooling similar EELS spectra also reduced the signal with respect to the inherent background noise.\\
The zero-loss peak subtraction applied by the means of a Batson correction was useful for increasing the contrast of the data at the high momentum transfer region and was able to remove the zero-loss peak. Keeping in mind that the Batson correction needs an image spectrum to reach its desired accuracy, which was not available, the result were sufficient for tracking the relatively obscure plasmon peak in the high momentum transfer regime.\\
The analysed EFTEM stack showed similar features to those found in earlier works looking into the band structure of Indium Selenide, the low momentum transfer q-EELS spectra showed dispersionless peaks at energy-loss values similar to those found before. The bulk plasmon peak starting at roughly $14eV$ energy loss was successfully tracked between diffraction spots and showed some anisotropic behaviours.\\
Overall the results agree with earlier implementations of the methods applied, the results would benefit from a higher energy resolution to give the methods more information to work with. The Python methods developed for this work will continued to be worked on.