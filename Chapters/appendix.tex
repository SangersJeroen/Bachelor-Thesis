\section*{Appendix}
Placeholder and code block test
\subsection*{Python}
The python code used to calculate the ellipse axes is displayed below.
\begin{lstlisting}[language=Python]
def ellipse_calc(x,y):
    # replace nans by average
    x_ = np.where(np.isnan(x), np.nanmean(x), x)
    y_ = np.where(np.isnan(y), np.nanmean(y), y)

    # Calculate variance and covariance
    var_x = np.sum((x_-np.mean(x_))**2)/(len(x)+1)
    var_y = np.sum((y_-np.mean(y_))**2)/(len(y)+1)
    cov = np.sum((x_-np.mean(x_))*(y_-np.mean(y_))/(len(x_)+1))

    cov_matrix = np.asarray([[var_x, cov],[cov,var_y]])
    evals,evecs = linalg.eig(cov_matrix)
    evecs_ = evals*evecs
    #plt.plot(x,y, linestyle='none',marker='x',zorder=1)
    #plt.quiver(np.nanmean(x),np.nanmean(y),-evecs_[1,:],-evecs_[0,:],
    zorder=2, units='xy', scale=1, width=1e-8, headwidth=4)


    a = np.max(evals)
    b = np.min(evals)
    print(a)
    print(b)
    index_a = np.where(evals == a)[0]
    theta = np.arctan(evecs[0,index_a]/evecs[1,index_a])
    print('theta =', theta)
    return a,b,theta
\end{lstlisting}






